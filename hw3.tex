\documentclass{article}

\usepackage[english]{babel}
\usepackage{lineno, blindtext}
\usepackage[utf8]{inputenc}
\usepackage{amsmath,amssymb}
\usepackage{parskip}
\usepackage{enumitem}
\usepackage{listings}
\usepackage{graphicx}
\graphicspath{ {./images/} }
\usepackage[ruled,vlined]{algorithm2e}
%\usepackage{algorithmic}
%\usepackage{algpseudocode}

% Margins
\usepackage[top=2.5cm, left=3cm, right=3cm, bottom=4.0cm]{geometry}
% Colour table cells
\usepackage[table]{xcolor}

% Get larger line spacing in table
\newcommand{\tablespace}{\\[1.25mm]}
\newcommand\Tstrut{\rule{0pt}{2.6ex}}         % = `top' strut
\newcommand\tstrut{\rule{0pt}{2.0ex}}         % = `top' strut
\newcommand\Bstrut{\rule[-0.9ex]{0pt}{0pt}}   % = `bottom' strut

%%%%%%%%%%%%%%%%%
%   Homework 1  %
%%%%%%%%%%%%%%%%%
\title{EE 382C Multicore Computing Homework 3}
\author{
    Mitchell, Olivia\\
    \texttt{ozm59}
    \and
    Molter, Matthew\\
    \texttt{mm58286}}
\date{\today}

\begin{document}
\maketitle

%%%%%%%%%%%%%%%%%
%   Problem 1   %
%%%%%%%%%%%%%%%%%
\section{Problem 1}
For each of the histories below, state whether it is
(a) sequentially consistent, (b) linearizable. Justify your answer.
All variables are initially zero.

\subsection{H1}
\begin{verbatim}
Concurrent History H1

P1           [ read(x) returns   1]
P2       [ write(x,1)                   ]    [ read(x)  returns 2]
P3             [write(x,2)      ]            
\end{verbatim}

\subsubsection{Solution part a - sequentially consistent?}
\subsubsection{Solution part b - linearizable?}

\subsection{H2}
\begin{verbatim}
Concurrent History H2

P1           [ read(x) returns   1]
P2       [ write(x,1)                   ]    [ read(x)  returns 1]
P3             [write(x,2)      ]            
\end{verbatim}

\subsubsection{Solution part a - sequentially consistent?}
\subsubsection{Solution part b - linearizable?}

\subsection{H3}
\begin{verbatim}
Concurrent History H3

P1           [ read(x) returns 1]
P2       [ write(x,1) ]                       [ read(x)  returns 1]
P3                        [write(x,2)      ]            
\end{verbatim}

\subsubsection{Solution part a - sequentially consistent?}
\subsubsection{Solution part b - linearizable?}

%%%%%%%%%%%%%%%%%
%   Problem 2   %
%%%%%%%%%%%%%%%%%
\pagebreak
\section{Problem 2}
Consider the following concurrent program.
\begin{verbatim}
Initially a, b and c are 0.
P1:  a:=1 ; print(b) ; print(c);
P2:  b:=1 ; print(a) ; print(c);
P3:  c:=1 ; print(a) ; print(b);
\end{verbatim}

Which of the outputs are sequentially consistent. Justify your answer.

(a) P1 outputs 11, P2 outputs 01 and P3 outputs 11.\\

(b) P1 outputs 00, P2 outputs 11 and P3 outputs 01.


\subsection{Solution 2a}


\subsection{Solution 2b}


%%%%%%%%%%%%%%%%%
%   Problem 3   %
%%%%%%%%%%%%%%%%%
\pagebreak
\section{Problem 3}
\subsection{part a - queues}
Implement Lock-based and Lock-Free unbounded queues of {\tt Integers}.
For the lock based implementation, use different locks for {\tt enq} and {\tt deq} operations.
For the variable {\tt count} use {\tt AtomicInteger}. 
For the lock-free implementation, 
use Michael and Scott's algorithm as
explained in the class. The {\em deq} operation should return {\em null}
if the queue is empty.\\

\subsection{part b - stacks}
Implement Lock-Free stack of {\tt Integer}. You should provide {\tt push(Integer x)}
and {\tt Integer pop()}. The {\tt pop} operation should throw an exception called
{\tt EmptyStack} if the stack is empty.
\\
For both the data structures use a list based implementation (rather than an array based implementation).

\end{document}
